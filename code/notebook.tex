
% Default to the notebook output style

    


% Inherit from the specified cell style.




    
\documentclass[11pt]{article}

    
    
    \usepackage[T1]{fontenc}
    % Nicer default font (+ math font) than Computer Modern for most use cases
    \usepackage{mathpazo}

    % Basic figure setup, for now with no caption control since it's done
    % automatically by Pandoc (which extracts ![](path) syntax from Markdown).
    \usepackage{graphicx}
    % We will generate all images so they have a width \maxwidth. This means
    % that they will get their normal width if they fit onto the page, but
    % are scaled down if they would overflow the margins.
    \makeatletter
    \def\maxwidth{\ifdim\Gin@nat@width>\linewidth\linewidth
    \else\Gin@nat@width\fi}
    \makeatother
    \let\Oldincludegraphics\includegraphics
    % Set max figure width to be 80% of text width, for now hardcoded.
    \renewcommand{\includegraphics}[1]{\Oldincludegraphics[width=.8\maxwidth]{#1}}
    % Ensure that by default, figures have no caption (until we provide a
    % proper Figure object with a Caption API and a way to capture that
    % in the conversion process - todo).
    \usepackage{caption}
    \DeclareCaptionLabelFormat{nolabel}{}
    \captionsetup{labelformat=nolabel}

    \usepackage{adjustbox} % Used to constrain images to a maximum size 
    \usepackage{xcolor} % Allow colors to be defined
    \usepackage{enumerate} % Needed for markdown enumerations to work
    \usepackage{geometry} % Used to adjust the document margins
    \usepackage{amsmath} % Equations
    \usepackage{amssymb} % Equations
    \usepackage{textcomp} % defines textquotesingle
    % Hack from http://tex.stackexchange.com/a/47451/13684:
    \AtBeginDocument{%
        \def\PYZsq{\textquotesingle}% Upright quotes in Pygmentized code
    }
    \usepackage{upquote} % Upright quotes for verbatim code
    \usepackage{eurosym} % defines \euro
    \usepackage[mathletters]{ucs} % Extended unicode (utf-8) support
    \usepackage[utf8x]{inputenc} % Allow utf-8 characters in the tex document
    \usepackage{fancyvrb} % verbatim replacement that allows latex
    \usepackage{grffile} % extends the file name processing of package graphics 
                         % to support a larger range 
    % The hyperref package gives us a pdf with properly built
    % internal navigation ('pdf bookmarks' for the table of contents,
    % internal cross-reference links, web links for URLs, etc.)
    \usepackage{hyperref}
    \usepackage{longtable} % longtable support required by pandoc >1.10
    \usepackage{booktabs}  % table support for pandoc > 1.12.2
    \usepackage[inline]{enumitem} % IRkernel/repr support (it uses the enumerate* environment)
    \usepackage[normalem]{ulem} % ulem is needed to support strikethroughs (\sout)
                                % normalem makes italics be italics, not underlines
    

    
    
    % Colors for the hyperref package
    \definecolor{urlcolor}{rgb}{0,.145,.698}
    \definecolor{linkcolor}{rgb}{.71,0.21,0.01}
    \definecolor{citecolor}{rgb}{.12,.54,.11}

    % ANSI colors
    \definecolor{ansi-black}{HTML}{3E424D}
    \definecolor{ansi-black-intense}{HTML}{282C36}
    \definecolor{ansi-red}{HTML}{E75C58}
    \definecolor{ansi-red-intense}{HTML}{B22B31}
    \definecolor{ansi-green}{HTML}{00A250}
    \definecolor{ansi-green-intense}{HTML}{007427}
    \definecolor{ansi-yellow}{HTML}{DDB62B}
    \definecolor{ansi-yellow-intense}{HTML}{B27D12}
    \definecolor{ansi-blue}{HTML}{208FFB}
    \definecolor{ansi-blue-intense}{HTML}{0065CA}
    \definecolor{ansi-magenta}{HTML}{D160C4}
    \definecolor{ansi-magenta-intense}{HTML}{A03196}
    \definecolor{ansi-cyan}{HTML}{60C6C8}
    \definecolor{ansi-cyan-intense}{HTML}{258F8F}
    \definecolor{ansi-white}{HTML}{C5C1B4}
    \definecolor{ansi-white-intense}{HTML}{A1A6B2}

    % commands and environments needed by pandoc snippets
    % extracted from the output of `pandoc -s`
    \providecommand{\tightlist}{%
      \setlength{\itemsep}{0pt}\setlength{\parskip}{0pt}}
    \DefineVerbatimEnvironment{Highlighting}{Verbatim}{commandchars=\\\{\}}
    % Add ',fontsize=\small' for more characters per line
    \newenvironment{Shaded}{}{}
    \newcommand{\KeywordTok}[1]{\textcolor[rgb]{0.00,0.44,0.13}{\textbf{{#1}}}}
    \newcommand{\DataTypeTok}[1]{\textcolor[rgb]{0.56,0.13,0.00}{{#1}}}
    \newcommand{\DecValTok}[1]{\textcolor[rgb]{0.25,0.63,0.44}{{#1}}}
    \newcommand{\BaseNTok}[1]{\textcolor[rgb]{0.25,0.63,0.44}{{#1}}}
    \newcommand{\FloatTok}[1]{\textcolor[rgb]{0.25,0.63,0.44}{{#1}}}
    \newcommand{\CharTok}[1]{\textcolor[rgb]{0.25,0.44,0.63}{{#1}}}
    \newcommand{\StringTok}[1]{\textcolor[rgb]{0.25,0.44,0.63}{{#1}}}
    \newcommand{\CommentTok}[1]{\textcolor[rgb]{0.38,0.63,0.69}{\textit{{#1}}}}
    \newcommand{\OtherTok}[1]{\textcolor[rgb]{0.00,0.44,0.13}{{#1}}}
    \newcommand{\AlertTok}[1]{\textcolor[rgb]{1.00,0.00,0.00}{\textbf{{#1}}}}
    \newcommand{\FunctionTok}[1]{\textcolor[rgb]{0.02,0.16,0.49}{{#1}}}
    \newcommand{\RegionMarkerTok}[1]{{#1}}
    \newcommand{\ErrorTok}[1]{\textcolor[rgb]{1.00,0.00,0.00}{\textbf{{#1}}}}
    \newcommand{\NormalTok}[1]{{#1}}
    
    % Additional commands for more recent versions of Pandoc
    \newcommand{\ConstantTok}[1]{\textcolor[rgb]{0.53,0.00,0.00}{{#1}}}
    \newcommand{\SpecialCharTok}[1]{\textcolor[rgb]{0.25,0.44,0.63}{{#1}}}
    \newcommand{\VerbatimStringTok}[1]{\textcolor[rgb]{0.25,0.44,0.63}{{#1}}}
    \newcommand{\SpecialStringTok}[1]{\textcolor[rgb]{0.73,0.40,0.53}{{#1}}}
    \newcommand{\ImportTok}[1]{{#1}}
    \newcommand{\DocumentationTok}[1]{\textcolor[rgb]{0.73,0.13,0.13}{\textit{{#1}}}}
    \newcommand{\AnnotationTok}[1]{\textcolor[rgb]{0.38,0.63,0.69}{\textbf{\textit{{#1}}}}}
    \newcommand{\CommentVarTok}[1]{\textcolor[rgb]{0.38,0.63,0.69}{\textbf{\textit{{#1}}}}}
    \newcommand{\VariableTok}[1]{\textcolor[rgb]{0.10,0.09,0.49}{{#1}}}
    \newcommand{\ControlFlowTok}[1]{\textcolor[rgb]{0.00,0.44,0.13}{\textbf{{#1}}}}
    \newcommand{\OperatorTok}[1]{\textcolor[rgb]{0.40,0.40,0.40}{{#1}}}
    \newcommand{\BuiltInTok}[1]{{#1}}
    \newcommand{\ExtensionTok}[1]{{#1}}
    \newcommand{\PreprocessorTok}[1]{\textcolor[rgb]{0.74,0.48,0.00}{{#1}}}
    \newcommand{\AttributeTok}[1]{\textcolor[rgb]{0.49,0.56,0.16}{{#1}}}
    \newcommand{\InformationTok}[1]{\textcolor[rgb]{0.38,0.63,0.69}{\textbf{\textit{{#1}}}}}
    \newcommand{\WarningTok}[1]{\textcolor[rgb]{0.38,0.63,0.69}{\textbf{\textit{{#1}}}}}
    
    
    % Define a nice break command that doesn't care if a line doesn't already
    % exist.
    \def\br{\hspace*{\fill} \\* }
    % Math Jax compatability definitions
    \def\gt{>}
    \def\lt{<}
    % Document parameters
    \title{Project2\_Final}
    
    
    

    % Pygments definitions
    
\makeatletter
\def\PY@reset{\let\PY@it=\relax \let\PY@bf=\relax%
    \let\PY@ul=\relax \let\PY@tc=\relax%
    \let\PY@bc=\relax \let\PY@ff=\relax}
\def\PY@tok#1{\csname PY@tok@#1\endcsname}
\def\PY@toks#1+{\ifx\relax#1\empty\else%
    \PY@tok{#1}\expandafter\PY@toks\fi}
\def\PY@do#1{\PY@bc{\PY@tc{\PY@ul{%
    \PY@it{\PY@bf{\PY@ff{#1}}}}}}}
\def\PY#1#2{\PY@reset\PY@toks#1+\relax+\PY@do{#2}}

\expandafter\def\csname PY@tok@w\endcsname{\def\PY@tc##1{\textcolor[rgb]{0.73,0.73,0.73}{##1}}}
\expandafter\def\csname PY@tok@c\endcsname{\let\PY@it=\textit\def\PY@tc##1{\textcolor[rgb]{0.25,0.50,0.50}{##1}}}
\expandafter\def\csname PY@tok@cp\endcsname{\def\PY@tc##1{\textcolor[rgb]{0.74,0.48,0.00}{##1}}}
\expandafter\def\csname PY@tok@k\endcsname{\let\PY@bf=\textbf\def\PY@tc##1{\textcolor[rgb]{0.00,0.50,0.00}{##1}}}
\expandafter\def\csname PY@tok@kp\endcsname{\def\PY@tc##1{\textcolor[rgb]{0.00,0.50,0.00}{##1}}}
\expandafter\def\csname PY@tok@kt\endcsname{\def\PY@tc##1{\textcolor[rgb]{0.69,0.00,0.25}{##1}}}
\expandafter\def\csname PY@tok@o\endcsname{\def\PY@tc##1{\textcolor[rgb]{0.40,0.40,0.40}{##1}}}
\expandafter\def\csname PY@tok@ow\endcsname{\let\PY@bf=\textbf\def\PY@tc##1{\textcolor[rgb]{0.67,0.13,1.00}{##1}}}
\expandafter\def\csname PY@tok@nb\endcsname{\def\PY@tc##1{\textcolor[rgb]{0.00,0.50,0.00}{##1}}}
\expandafter\def\csname PY@tok@nf\endcsname{\def\PY@tc##1{\textcolor[rgb]{0.00,0.00,1.00}{##1}}}
\expandafter\def\csname PY@tok@nc\endcsname{\let\PY@bf=\textbf\def\PY@tc##1{\textcolor[rgb]{0.00,0.00,1.00}{##1}}}
\expandafter\def\csname PY@tok@nn\endcsname{\let\PY@bf=\textbf\def\PY@tc##1{\textcolor[rgb]{0.00,0.00,1.00}{##1}}}
\expandafter\def\csname PY@tok@ne\endcsname{\let\PY@bf=\textbf\def\PY@tc##1{\textcolor[rgb]{0.82,0.25,0.23}{##1}}}
\expandafter\def\csname PY@tok@nv\endcsname{\def\PY@tc##1{\textcolor[rgb]{0.10,0.09,0.49}{##1}}}
\expandafter\def\csname PY@tok@no\endcsname{\def\PY@tc##1{\textcolor[rgb]{0.53,0.00,0.00}{##1}}}
\expandafter\def\csname PY@tok@nl\endcsname{\def\PY@tc##1{\textcolor[rgb]{0.63,0.63,0.00}{##1}}}
\expandafter\def\csname PY@tok@ni\endcsname{\let\PY@bf=\textbf\def\PY@tc##1{\textcolor[rgb]{0.60,0.60,0.60}{##1}}}
\expandafter\def\csname PY@tok@na\endcsname{\def\PY@tc##1{\textcolor[rgb]{0.49,0.56,0.16}{##1}}}
\expandafter\def\csname PY@tok@nt\endcsname{\let\PY@bf=\textbf\def\PY@tc##1{\textcolor[rgb]{0.00,0.50,0.00}{##1}}}
\expandafter\def\csname PY@tok@nd\endcsname{\def\PY@tc##1{\textcolor[rgb]{0.67,0.13,1.00}{##1}}}
\expandafter\def\csname PY@tok@s\endcsname{\def\PY@tc##1{\textcolor[rgb]{0.73,0.13,0.13}{##1}}}
\expandafter\def\csname PY@tok@sd\endcsname{\let\PY@it=\textit\def\PY@tc##1{\textcolor[rgb]{0.73,0.13,0.13}{##1}}}
\expandafter\def\csname PY@tok@si\endcsname{\let\PY@bf=\textbf\def\PY@tc##1{\textcolor[rgb]{0.73,0.40,0.53}{##1}}}
\expandafter\def\csname PY@tok@se\endcsname{\let\PY@bf=\textbf\def\PY@tc##1{\textcolor[rgb]{0.73,0.40,0.13}{##1}}}
\expandafter\def\csname PY@tok@sr\endcsname{\def\PY@tc##1{\textcolor[rgb]{0.73,0.40,0.53}{##1}}}
\expandafter\def\csname PY@tok@ss\endcsname{\def\PY@tc##1{\textcolor[rgb]{0.10,0.09,0.49}{##1}}}
\expandafter\def\csname PY@tok@sx\endcsname{\def\PY@tc##1{\textcolor[rgb]{0.00,0.50,0.00}{##1}}}
\expandafter\def\csname PY@tok@m\endcsname{\def\PY@tc##1{\textcolor[rgb]{0.40,0.40,0.40}{##1}}}
\expandafter\def\csname PY@tok@gh\endcsname{\let\PY@bf=\textbf\def\PY@tc##1{\textcolor[rgb]{0.00,0.00,0.50}{##1}}}
\expandafter\def\csname PY@tok@gu\endcsname{\let\PY@bf=\textbf\def\PY@tc##1{\textcolor[rgb]{0.50,0.00,0.50}{##1}}}
\expandafter\def\csname PY@tok@gd\endcsname{\def\PY@tc##1{\textcolor[rgb]{0.63,0.00,0.00}{##1}}}
\expandafter\def\csname PY@tok@gi\endcsname{\def\PY@tc##1{\textcolor[rgb]{0.00,0.63,0.00}{##1}}}
\expandafter\def\csname PY@tok@gr\endcsname{\def\PY@tc##1{\textcolor[rgb]{1.00,0.00,0.00}{##1}}}
\expandafter\def\csname PY@tok@ge\endcsname{\let\PY@it=\textit}
\expandafter\def\csname PY@tok@gs\endcsname{\let\PY@bf=\textbf}
\expandafter\def\csname PY@tok@gp\endcsname{\let\PY@bf=\textbf\def\PY@tc##1{\textcolor[rgb]{0.00,0.00,0.50}{##1}}}
\expandafter\def\csname PY@tok@go\endcsname{\def\PY@tc##1{\textcolor[rgb]{0.53,0.53,0.53}{##1}}}
\expandafter\def\csname PY@tok@gt\endcsname{\def\PY@tc##1{\textcolor[rgb]{0.00,0.27,0.87}{##1}}}
\expandafter\def\csname PY@tok@err\endcsname{\def\PY@bc##1{\setlength{\fboxsep}{0pt}\fcolorbox[rgb]{1.00,0.00,0.00}{1,1,1}{\strut ##1}}}
\expandafter\def\csname PY@tok@kc\endcsname{\let\PY@bf=\textbf\def\PY@tc##1{\textcolor[rgb]{0.00,0.50,0.00}{##1}}}
\expandafter\def\csname PY@tok@kd\endcsname{\let\PY@bf=\textbf\def\PY@tc##1{\textcolor[rgb]{0.00,0.50,0.00}{##1}}}
\expandafter\def\csname PY@tok@kn\endcsname{\let\PY@bf=\textbf\def\PY@tc##1{\textcolor[rgb]{0.00,0.50,0.00}{##1}}}
\expandafter\def\csname PY@tok@kr\endcsname{\let\PY@bf=\textbf\def\PY@tc##1{\textcolor[rgb]{0.00,0.50,0.00}{##1}}}
\expandafter\def\csname PY@tok@bp\endcsname{\def\PY@tc##1{\textcolor[rgb]{0.00,0.50,0.00}{##1}}}
\expandafter\def\csname PY@tok@fm\endcsname{\def\PY@tc##1{\textcolor[rgb]{0.00,0.00,1.00}{##1}}}
\expandafter\def\csname PY@tok@vc\endcsname{\def\PY@tc##1{\textcolor[rgb]{0.10,0.09,0.49}{##1}}}
\expandafter\def\csname PY@tok@vg\endcsname{\def\PY@tc##1{\textcolor[rgb]{0.10,0.09,0.49}{##1}}}
\expandafter\def\csname PY@tok@vi\endcsname{\def\PY@tc##1{\textcolor[rgb]{0.10,0.09,0.49}{##1}}}
\expandafter\def\csname PY@tok@vm\endcsname{\def\PY@tc##1{\textcolor[rgb]{0.10,0.09,0.49}{##1}}}
\expandafter\def\csname PY@tok@sa\endcsname{\def\PY@tc##1{\textcolor[rgb]{0.73,0.13,0.13}{##1}}}
\expandafter\def\csname PY@tok@sb\endcsname{\def\PY@tc##1{\textcolor[rgb]{0.73,0.13,0.13}{##1}}}
\expandafter\def\csname PY@tok@sc\endcsname{\def\PY@tc##1{\textcolor[rgb]{0.73,0.13,0.13}{##1}}}
\expandafter\def\csname PY@tok@dl\endcsname{\def\PY@tc##1{\textcolor[rgb]{0.73,0.13,0.13}{##1}}}
\expandafter\def\csname PY@tok@s2\endcsname{\def\PY@tc##1{\textcolor[rgb]{0.73,0.13,0.13}{##1}}}
\expandafter\def\csname PY@tok@sh\endcsname{\def\PY@tc##1{\textcolor[rgb]{0.73,0.13,0.13}{##1}}}
\expandafter\def\csname PY@tok@s1\endcsname{\def\PY@tc##1{\textcolor[rgb]{0.73,0.13,0.13}{##1}}}
\expandafter\def\csname PY@tok@mb\endcsname{\def\PY@tc##1{\textcolor[rgb]{0.40,0.40,0.40}{##1}}}
\expandafter\def\csname PY@tok@mf\endcsname{\def\PY@tc##1{\textcolor[rgb]{0.40,0.40,0.40}{##1}}}
\expandafter\def\csname PY@tok@mh\endcsname{\def\PY@tc##1{\textcolor[rgb]{0.40,0.40,0.40}{##1}}}
\expandafter\def\csname PY@tok@mi\endcsname{\def\PY@tc##1{\textcolor[rgb]{0.40,0.40,0.40}{##1}}}
\expandafter\def\csname PY@tok@il\endcsname{\def\PY@tc##1{\textcolor[rgb]{0.40,0.40,0.40}{##1}}}
\expandafter\def\csname PY@tok@mo\endcsname{\def\PY@tc##1{\textcolor[rgb]{0.40,0.40,0.40}{##1}}}
\expandafter\def\csname PY@tok@ch\endcsname{\let\PY@it=\textit\def\PY@tc##1{\textcolor[rgb]{0.25,0.50,0.50}{##1}}}
\expandafter\def\csname PY@tok@cm\endcsname{\let\PY@it=\textit\def\PY@tc##1{\textcolor[rgb]{0.25,0.50,0.50}{##1}}}
\expandafter\def\csname PY@tok@cpf\endcsname{\let\PY@it=\textit\def\PY@tc##1{\textcolor[rgb]{0.25,0.50,0.50}{##1}}}
\expandafter\def\csname PY@tok@c1\endcsname{\let\PY@it=\textit\def\PY@tc##1{\textcolor[rgb]{0.25,0.50,0.50}{##1}}}
\expandafter\def\csname PY@tok@cs\endcsname{\let\PY@it=\textit\def\PY@tc##1{\textcolor[rgb]{0.25,0.50,0.50}{##1}}}

\def\PYZbs{\char`\\}
\def\PYZus{\char`\_}
\def\PYZob{\char`\{}
\def\PYZcb{\char`\}}
\def\PYZca{\char`\^}
\def\PYZam{\char`\&}
\def\PYZlt{\char`\<}
\def\PYZgt{\char`\>}
\def\PYZsh{\char`\#}
\def\PYZpc{\char`\%}
\def\PYZdl{\char`\$}
\def\PYZhy{\char`\-}
\def\PYZsq{\char`\'}
\def\PYZdq{\char`\"}
\def\PYZti{\char`\~}
% for compatibility with earlier versions
\def\PYZat{@}
\def\PYZlb{[}
\def\PYZrb{]}
\makeatother


    % Exact colors from NB
    \definecolor{incolor}{rgb}{0.0, 0.0, 0.5}
    \definecolor{outcolor}{rgb}{0.545, 0.0, 0.0}



    
    % Prevent overflowing lines due to hard-to-break entities
    \sloppy 
    % Setup hyperref package
    \hypersetup{
      breaklinks=true,  % so long urls are correctly broken across lines
      colorlinks=true,
      urlcolor=urlcolor,
      linkcolor=linkcolor,
      citecolor=citecolor,
      }
    % Slightly bigger margins than the latex defaults
    
    \geometry{verbose,tmargin=1in,bmargin=1in,lmargin=1in,rmargin=1in}
    
    

    \begin{document}
    
    
    \maketitle
    
    

    
    \hypertarget{project-2-team-11}{%
\section{PROJECT 2 : TEAM 11}\label{project-2-team-11}}

    Members: Talia Tandler, SeungU Lyu

    \begin{Verbatim}[commandchars=\\\{\}]
{\color{incolor}In [{\color{incolor}61}]:} \PY{c+c1}{\PYZsh{} Configure Jupyter so figures appear in the notebook}
         \PY{o}{\PYZpc{}}\PY{k}{matplotlib} inline
         
         \PY{c+c1}{\PYZsh{} Configure Jupyter to display the assigned value after an assignment}
         \PY{o}{\PYZpc{}}\PY{k}{config} InteractiveShell.ast\PYZus{}node\PYZus{}interactivity=\PYZsq{}last\PYZus{}expr\PYZus{}or\PYZus{}assign\PYZsq{}
         
         \PY{c+c1}{\PYZsh{} import functions from the modsim.py module}
         \PY{k+kn}{from} \PY{n+nn}{modsim} \PY{k}{import} \PY{o}{*}
         \PY{k+kn}{import} \PY{n+nn}{math}
\end{Verbatim}


    http://www.worldometers.info/world-population/us-population/ US pop in
2017 = 324,459,463

https://wwwnc.cdc.gov/travel/yellowbook/2018/infectious-diseases-related-to-travel/measles-rubeola
Measles incubation period 11 days average, infectious period 2-4 days
before rash to after rash.

https://www.cdc.gov/vaccines/imz-managers/coverage/childvaxview/data-reports/mmr/trend/index.html
MMR immunization rate in 2017 = 90.7\%

    \begin{Verbatim}[commandchars=\\\{\}]
{\color{incolor}In [{\color{incolor}62}]:} \PY{c+c1}{\PYZsh{}population}
         \PY{n}{pop} \PY{o}{=} \PY{l+m+mi}{999}
         
         \PY{c+c1}{\PYZsh{}initial immunity of the US population}
         \PY{n}{init\PYZus{}im} \PY{o}{=} \PY{l+m+mf}{0.907}    
         
         \PY{c+c1}{\PYZsh{}assumed contact rate}
         \PY{n}{beta} \PY{o}{=} \PY{l+m+mf}{0.9}
         
         \PY{c+c1}{\PYZsh{}US recovery rate from measles}
         \PY{n}{gamma} \PY{o}{=} \PY{l+m+mi}{1}\PY{o}{/}\PY{l+m+mi}{7}
         
         \PY{c+c1}{\PYZsh{}US rate from exposure period of 11 days to infected}
         \PY{n}{sigma} \PY{o}{=} \PY{l+m+mf}{0.091}\PY{p}{;}     
\end{Verbatim}


    \hypertarget{question}{%
\subsection{Question}\label{question}}

\hypertarget{what-is-the-result-of-lowering-the-measles-immunity-rate-in-a-small-community-during-a-outbreak}{%
\subsubsection{What is the result of lowering the measles immunity rate
in a small community during a
outbreak?}\label{what-is-the-result-of-lowering-the-measles-immunity-rate-in-a-small-community-during-a-outbreak}}

Measles is a highly infectious disease that can infect about 90\% of
people that come into contact with the patient. However, the disease is
not common these days because of the MMR vaccination, which can
effectively prevent people getting the disease. Due to the high
vaccination rate, the United States was declared free of circulating
measles in 2000. However there were 911 cases of measles between 2001
and 2011. These occurences arose due to individuals from other countries
entering the U.S. with measles. Because of the disease's high infectious
rate upon contact, herd immunity is considered very important for
measles.

In 2015, a measles outbreak occured at Disney World causing more than
159 people to be infected during a single outbreak. Only
50\textasciitilde{}86\% people exposed to this outbreak were vaccinated,
causing an even bigger outbreak. This vaccination was lower than it
should have been due to Anti-Vaccination Movements in the U.S. These
lower rates lowered the population immunity rate and caused the herd
immunity to not function as expected. The starter of this movement,
Andrew Wakefield, stated that the MMR vaccination can cause autism in
newborn children because of the mercury content inside the specific
vaccine. Due to this false research, many parents became concerned with
the side effects of the vaccination and opted to not vaccinate their
children with MMR. As a result, there was a decently sized generation of
children susceptible to measles because they did not receive the
vaccination at birth.

This simulation utilizes an SEIR model to understand how varying the
measles immunity rate in a community effects herd immunity.

\hypertarget{methodology}{%
\subsection{Methodology}\label{methodology}}

In order to create this model, we:

\begin{enumerate}
\def\labelenumi{\arabic{enumi}.}
\tightlist
\item
  Did background research on the MMR vaccination and the measles
  diseases and found a set of constants we would implement in our model.
\item
  Put the variables into a state function.
\item
  Set the total population to 1000, with initial infection number as one
  person infected with measles.
\item
  Ran the simulation based on the number measles infections every day.
\item
  Set a condition where the measles outbreak ends when the number
  infected people is less than one person.
\item
  Created graphs to visually represent our results.
\end{enumerate}

    \begin{Verbatim}[commandchars=\\\{\}]
{\color{incolor}In [{\color{incolor}64}]:} \PY{k}{def} \PY{n+nf}{make\PYZus{}system} \PY{p}{(}\PY{n}{pop}\PY{p}{,} \PY{n}{init\PYZus{}im}\PY{p}{,} \PY{n}{beta}\PY{p}{,} \PY{n}{gamma}\PY{p}{,} \PY{n}{sigma}\PY{p}{)}\PY{p}{:}
             \PY{l+s+sd}{\PYZdq{}\PYZdq{}\PYZdq{}Make a system object for the SCIR model}
         \PY{l+s+sd}{    }
         \PY{l+s+sd}{    pop: Total US population}
         \PY{l+s+sd}{    init\PYZus{}im: Initial Population Immunity}
         \PY{l+s+sd}{    beta: effective contact number for patient}
         \PY{l+s+sd}{    gamma: recovery rate for infected people}
         \PY{l+s+sd}{    sigma: rate of incubation group moving to infectious group}
         \PY{l+s+sd}{    return: System object\PYZdq{}\PYZdq{}\PYZdq{}}
             
             \PY{c+c1}{\PYZsh{}S: susceptible, E: exposed period, I: infected, R: recovered(immune to disease)}
             \PY{n}{init} \PY{o}{=} \PY{n}{State}\PY{p}{(}\PY{n}{S} \PY{o}{=} \PY{n+nb}{int}\PY{p}{(}\PY{n}{pop}\PY{o}{*}\PY{p}{(}\PY{l+m+mi}{1} \PY{o}{\PYZhy{}} \PY{n}{init\PYZus{}im}\PY{p}{)}\PY{p}{)}\PY{p}{,} \PY{n}{E} \PY{o}{=} \PY{l+m+mi}{0}\PY{p}{,} \PY{n}{I} \PY{o}{=} \PY{l+m+mi}{1}\PY{p}{,} \PY{n}{R} \PY{o}{=} \PY{n+nb}{int}\PY{p}{(}\PY{n}{pop}\PY{o}{*}\PY{n}{init\PYZus{}im}\PY{p}{)}\PY{p}{)}
             \PY{n}{init} \PY{o}{/}\PY{o}{=} \PY{n}{np}\PY{o}{.}\PY{n}{sum}\PY{p}{(}\PY{n}{init}\PY{p}{)}
             
             \PY{n}{t0} \PY{o}{=} \PY{l+m+mi}{0}
             \PY{c+c1}{\PYZsh{}number of days in 1 year}
             \PY{n}{t\PYZus{}end} \PY{o}{=} \PY{l+m+mi}{365} 
             
             \PY{k}{return} \PY{n}{System}\PY{p}{(}\PY{n}{init} \PY{o}{=} \PY{n}{init}\PY{p}{,}
                           \PY{n}{beta} \PY{o}{=} \PY{n}{beta}\PY{p}{,}
                           \PY{n}{gamma} \PY{o}{=} \PY{n}{gamma}\PY{p}{,}
                           \PY{n}{sigma} \PY{o}{=} \PY{n}{sigma}\PY{p}{,}
                           \PY{n}{t0} \PY{o}{=} \PY{n}{t0}\PY{p}{,}
                           \PY{n}{t\PYZus{}end} \PY{o}{=} \PY{n}{t\PYZus{}end}\PY{p}{,}
                           \PY{n}{init\PYZus{}im} \PY{o}{=} \PY{n}{init\PYZus{}im}\PY{p}{)}
\end{Verbatim}


    make\_system function sets the initial values for the state and returns
it with other necessary variables. Since the model is a SEIR model,
initial state init contains four values, S, E, I, R where S and R is
determined by the initial size and immunization rate of the community,
and I is set to 1 to show that one person is infected at the start. Time
span for the simulation was set to a year, since every outbreak in this
simulation ends within the period.

    \begin{Verbatim}[commandchars=\\\{\}]
{\color{incolor}In [{\color{incolor}65}]:} \PY{k}{def} \PY{n+nf}{update\PYZus{}func}\PY{p}{(}\PY{n}{state}\PY{p}{,} \PY{n}{time}\PY{p}{,} \PY{n}{system}\PY{p}{)}\PY{p}{:}
             \PY{l+s+sd}{\PYZdq{}\PYZdq{}\PYZdq{}Update the SEIR model}
         \PY{l+s+sd}{    }
         \PY{l+s+sd}{    state: starting variables of SEIR}
         \PY{l+s+sd}{    t: time step}
         \PY{l+s+sd}{    system: includes alpha,beta,gamma,omega rates}
         \PY{l+s+sd}{    contact: current contact number for the state}
         \PY{l+s+sd}{    }
         \PY{l+s+sd}{    \PYZdq{}\PYZdq{}\PYZdq{}}
             
             \PY{n}{unpack}\PY{p}{(}\PY{n}{system}\PY{p}{)}
             
             \PY{n}{s}\PY{p}{,}\PY{n}{e}\PY{p}{,}\PY{n}{i}\PY{p}{,}\PY{n}{r} \PY{o}{=} \PY{n}{state}
                     
             \PY{c+c1}{\PYZsh{}current population}
             \PY{n}{total\PYZus{}pop} \PY{o}{=} \PY{n}{s}\PY{o}{+}\PY{n}{e}\PY{o}{+}\PY{n}{i}\PY{o}{+}\PY{n}{r}
             
             \PY{c+c1}{\PYZsh{}change rate for each status}
             
             \PY{c+c1}{\PYZsh{}change in number of people susceptible}
             \PY{n}{ds} \PY{o}{=} \PY{p}{(}\PY{o}{\PYZhy{}}\PY{n}{beta}\PY{o}{*}\PY{n}{s}\PY{o}{*}\PY{n}{i}\PY{p}{)}\PY{o}{/}\PY{n}{total\PYZus{}pop} 
             \PY{c+c1}{\PYZsh{}change in number of people moving to exposed period}
             \PY{n}{de} \PY{o}{=} \PY{p}{(}\PY{p}{(}\PY{n}{beta}\PY{o}{*}\PY{n}{s}\PY{o}{*}\PY{n}{i}\PY{p}{)}\PY{o}{/}\PY{n}{total\PYZus{}pop}\PY{p}{)} \PY{o}{\PYZhy{}} \PY{n}{sigma}\PY{o}{*}\PY{n}{e} 
             \PY{c+c1}{\PYZsh{}change in people moving to infectious period}
             \PY{n}{di} \PY{o}{=} \PY{n}{sigma}\PY{o}{*}\PY{n}{e} \PY{o}{\PYZhy{}} \PY{n}{gamma}\PY{o}{*}\PY{n}{i}     
             \PY{c+c1}{\PYZsh{}change in people recovered}
             \PY{n}{dr} \PY{o}{=} \PY{n}{gamma}\PY{o}{*}\PY{n}{i}     
         
             \PY{n}{s} \PY{o}{+}\PY{o}{=} \PY{n}{ds}      
             \PY{n}{e} \PY{o}{+}\PY{o}{=} \PY{n}{de}      
             \PY{n}{i} \PY{o}{+}\PY{o}{=} \PY{n}{di}      
             \PY{n}{r} \PY{o}{+}\PY{o}{=} \PY{n}{dr}      
             
             \PY{k}{return} \PY{n}{State}\PY{p}{(}\PY{n}{S}\PY{o}{=}\PY{n}{s}\PY{p}{,} \PY{n}{E}\PY{o}{=}\PY{n}{e}\PY{p}{,} \PY{n}{I}\PY{o}{=}\PY{n}{i}\PY{p}{,} \PY{n}{R}\PY{o}{=}\PY{n}{r}\PY{p}{)}
\end{Verbatim}


    update\_func function updates the state with four different differential
equations. System object was unpacked at the beginning of the code to
make it easy to read. Change in susceptible group is affected only by
the number of people in infected group, which will raise the number of
people in exposed group. There is no direct transition from susceptible
group to the infected group, because measles have average of 11 days
incubation period, where the person does not spread the disease during
that period. Therefore, about 1/11 (sigma value) of people in the
exposed group move to the infected group every day, showing that their
incubatoin period has ended. It takes about 7 days in average for people
to get recoverd, so 1/7 (gamma) of people infected is recovered every
day.

    \begin{Verbatim}[commandchars=\\\{\}]
{\color{incolor}In [{\color{incolor}66}]:} \PY{k}{def} \PY{n+nf}{run\PYZus{}simulation}\PY{p}{(}\PY{n}{system}\PY{p}{,} \PY{n}{update\PYZus{}func}\PY{p}{)}\PY{p}{:}
             \PY{l+s+sd}{\PYZdq{}\PYZdq{}\PYZdq{}Runs a simulation of the system.}
         \PY{l+s+sd}{        }
         \PY{l+s+sd}{    system: System object}
         \PY{l+s+sd}{    update\PYZus{}func: function that updates state}
         \PY{l+s+sd}{    }
         \PY{l+s+sd}{    returns: TimeFrame}
         \PY{l+s+sd}{    \PYZdq{}\PYZdq{}\PYZdq{}}
             \PY{n}{unpack}\PY{p}{(}\PY{n}{system}\PY{p}{)}
             
             \PY{c+c1}{\PYZsh{}creates timeframe to save daily states}
             \PY{n}{frame} \PY{o}{=} \PY{n}{TimeFrame}\PY{p}{(}\PY{n}{columns}\PY{o}{=}\PY{n}{init}\PY{o}{.}\PY{n}{index}\PY{p}{)}
             \PY{n}{frame}\PY{o}{.}\PY{n}{row}\PY{p}{[}\PY{n}{t0}\PY{p}{]} \PY{o}{=} \PY{n}{init}
             
             \PY{k}{for} \PY{n}{time} \PY{o+ow}{in} \PY{n}{linrange}\PY{p}{(}\PY{n}{t0}\PY{p}{,} \PY{n}{t\PYZus{}end}\PY{p}{)}\PY{p}{:}
                 \PY{n}{frame}\PY{o}{.}\PY{n}{row}\PY{p}{[}\PY{n}{time}\PY{o}{+}\PY{l+m+mi}{1}\PY{p}{]} \PY{o}{=} \PY{n}{update\PYZus{}func}\PY{p}{(}\PY{n}{frame}\PY{o}{.}\PY{n}{row}\PY{p}{[}\PY{n}{time}\PY{p}{]}\PY{p}{,} \PY{n}{time}\PY{p}{,} \PY{n}{system}\PY{p}{)}
             
             \PY{k}{return} \PY{n}{frame}
\end{Verbatim}


    run\_simulation function takes a system object with a update\_func
function, and simulates the state for the duration of the time span set
at the make\_system function. It returns a TimeFrame object with all the
state values for each time step.

    \begin{Verbatim}[commandchars=\\\{\}]
{\color{incolor}In [{\color{incolor}67}]:} \PY{k}{def} \PY{n+nf}{plot\PYZus{}results} \PY{p}{(}\PY{n}{S}\PY{p}{,}\PY{n}{E}\PY{p}{,}\PY{n}{I}\PY{p}{,}\PY{n}{R}\PY{p}{)}\PY{p}{:}
             
             \PY{n}{plot}\PY{p}{(}\PY{n}{S}\PY{p}{,} \PY{l+s+s1}{\PYZsq{}}\PY{l+s+s1}{\PYZhy{}\PYZhy{}}\PY{l+s+s1}{\PYZsq{}}\PY{p}{,} \PY{n}{label} \PY{o}{=} \PY{l+s+s1}{\PYZsq{}}\PY{l+s+s1}{Susceptible}\PY{l+s+s1}{\PYZsq{}}\PY{p}{)}
             \PY{n}{plot}\PY{p}{(}\PY{n}{E}\PY{p}{,} \PY{l+s+s1}{\PYZsq{}}\PY{l+s+s1}{\PYZhy{}}\PY{l+s+s1}{\PYZsq{}}\PY{p}{,} \PY{n}{label} \PY{o}{=} \PY{l+s+s1}{\PYZsq{}}\PY{l+s+s1}{Exposed}\PY{l+s+s1}{\PYZsq{}}\PY{p}{)}
             \PY{n}{plot}\PY{p}{(}\PY{n}{I}\PY{p}{,} \PY{l+s+s1}{\PYZsq{}}\PY{l+s+s1}{.}\PY{l+s+s1}{\PYZsq{}}\PY{p}{,} \PY{n}{label} \PY{o}{=} \PY{l+s+s1}{\PYZsq{}}\PY{l+s+s1}{Infected}\PY{l+s+s1}{\PYZsq{}}\PY{p}{)}
             \PY{n}{plot}\PY{p}{(}\PY{n}{R}\PY{p}{,} \PY{l+s+s1}{\PYZsq{}}\PY{l+s+s1}{:}\PY{l+s+s1}{\PYZsq{}}\PY{p}{,} \PY{n}{label} \PY{o}{=} \PY{l+s+s1}{\PYZsq{}}\PY{l+s+s1}{Recovered}\PY{l+s+s1}{\PYZsq{}}\PY{p}{)}
             \PY{n}{decorate}\PY{p}{(}\PY{n}{xlabel}\PY{o}{=}\PY{l+s+s1}{\PYZsq{}}\PY{l+s+s1}{Time (days)}\PY{l+s+s1}{\PYZsq{}}\PY{p}{,}
                     \PY{n}{ylabel} \PY{o}{=} \PY{l+s+s1}{\PYZsq{}}\PY{l+s+s1}{Fraction of population}\PY{l+s+s1}{\PYZsq{}}\PY{p}{)}
\end{Verbatim}


    A plotting function was made for convenience.

    \begin{Verbatim}[commandchars=\\\{\}]
{\color{incolor}In [{\color{incolor}69}]:} \PY{n}{init\PYZus{}im} \PY{o}{=} \PY{l+m+mf}{0.907}
         \PY{n}{system} \PY{o}{=} \PY{n}{make\PYZus{}system}\PY{p}{(}\PY{n}{pop}\PY{p}{,} \PY{n}{init\PYZus{}im}\PY{p}{,} \PY{n}{beta}\PY{p}{,} \PY{n}{gamma}\PY{p}{,} \PY{n}{sigma}\PY{p}{)}
         \PY{n}{results} \PY{o}{=} \PY{n}{run\PYZus{}simulation}\PY{p}{(}\PY{n}{system}\PY{p}{,} \PY{n}{update\PYZus{}func}\PY{p}{)}\PY{p}{;}
\end{Verbatim}


    The code was tested with 2017 average immunization rate for the U.S
(90.7\%), testing out what will happen if a measles infected person is
introduced to a community of 1000 people in a real world situation.

    \begin{Verbatim}[commandchars=\\\{\}]
{\color{incolor}In [{\color{incolor}70}]:} \PY{n}{plot\PYZus{}results}\PY{p}{(}\PY{n}{results}\PY{o}{.}\PY{n}{S}\PY{p}{,} \PY{n}{results}\PY{o}{.}\PY{n}{E}\PY{p}{,} \PY{n}{results}\PY{o}{.}\PY{n}{I}\PY{p}{,} \PY{n}{results}\PY{o}{.}\PY{n}{R}\PY{p}{)}
         \PY{n}{decorate}\PY{p}{(}\PY{n}{title} \PY{o}{=}\PY{l+s+s1}{\PYZsq{}}\PY{l+s+s1}{Figure 1}\PY{l+s+s1}{\PYZsq{}}\PY{p}{)}
\end{Verbatim}


    \begin{center}
    \adjustimage{max size={0.9\linewidth}{0.9\paperheight}}{output_16_0.png}
    \end{center}
    { \hspace*{\fill} \\}
    
    The result shows that even though measles is a highly contagious
disease, the measles outbreak ends without infecting number of people
due high immunity rate. We call this herd immunity, because immunized
people acts as a barrier that prevents disease to spread among the
susceptible people. For each disease, there is specific percentage of
people needed to create a herd immunity. Lowering the immunity rate will
show abrupt change in infected people, once the herd immunity stops
working.

    \begin{Verbatim}[commandchars=\\\{\}]
{\color{incolor}In [{\color{incolor}71}]:} \PY{n}{init\PYZus{}im2} \PY{o}{=} \PY{l+m+mf}{0.3}
         \PY{n}{system} \PY{o}{=} \PY{n}{make\PYZus{}system}\PY{p}{(}\PY{n}{pop}\PY{p}{,} \PY{n}{init\PYZus{}im2}\PY{p}{,} \PY{n}{beta}\PY{p}{,} \PY{n}{gamma}\PY{p}{,} \PY{n}{sigma}\PY{p}{)}
         \PY{n}{results2} \PY{o}{=} \PY{n}{run\PYZus{}simulation}\PY{p}{(}\PY{n}{system}\PY{p}{,} \PY{n}{update\PYZus{}func}\PY{p}{)}
         \PY{n}{results2}\PY{p}{;}
\end{Verbatim}


    Next, the code was tested with lowered initial immunity rate of 30\%.

    \begin{Verbatim}[commandchars=\\\{\}]
{\color{incolor}In [{\color{incolor}72}]:} \PY{n}{plot\PYZus{}results}\PY{p}{(}\PY{n}{results2}\PY{o}{.}\PY{n}{S}\PY{p}{,} \PY{n}{results2}\PY{o}{.}\PY{n}{E}\PY{p}{,} \PY{n}{results2}\PY{o}{.}\PY{n}{I}\PY{p}{,} \PY{n}{results2}\PY{o}{.}\PY{n}{R}\PY{p}{)}
         \PY{n}{decorate} \PY{p}{(}\PY{n}{title} \PY{o}{=} \PY{l+s+s1}{\PYZsq{}}\PY{l+s+s1}{Figure 2}\PY{l+s+s1}{\PYZsq{}}\PY{p}{)}
\end{Verbatim}


    \begin{center}
    \adjustimage{max size={0.9\linewidth}{0.9\paperheight}}{output_20_0.png}
    \end{center}
    { \hspace*{\fill} \\}
    
    The result is a lot different from the one above, showing that most of
susceptible people become infected before the outbreak ends. This shows
that the community with only 30\% immunity rate has lost their herd
immunity, because the number of immuned (recovered) people is too small
to act as a barrier that protects the susceptible people. Seeing the
result, we can assume that there must be a point between immunity rate
of 30\% to 90\% where the herd immunity fails to function.

    \begin{Verbatim}[commandchars=\\\{\}]
{\color{incolor}In [{\color{incolor}73}]:} \PY{k}{def} \PY{n+nf}{calc\PYZus{}highest\PYZus{}infected}\PY{p}{(}\PY{n}{results}\PY{p}{)}\PY{p}{:}
             \PY{l+s+sd}{\PYZdq{}\PYZdq{}\PYZdq{}Fraction of population infected during the simulation.}
         \PY{l+s+sd}{    }
         \PY{l+s+sd}{    results: DataFrame with columns S, E, I, R}
         \PY{l+s+sd}{    }
         \PY{l+s+sd}{    returns: fraction of population}
         \PY{l+s+sd}{    \PYZdq{}\PYZdq{}\PYZdq{}}
             \PY{k}{return} \PY{n+nb}{max}\PY{p}{(}\PY{n}{results}\PY{o}{.}\PY{n}{I}\PY{p}{)}
\end{Verbatim}


    \begin{Verbatim}[commandchars=\\\{\}]
{\color{incolor}In [{\color{incolor}74}]:} \PY{k}{def} \PY{n+nf}{sweep\PYZus{}init\PYZus{}im}\PY{p}{(}\PY{n}{imun\PYZus{}rate\PYZus{}array}\PY{p}{)}\PY{p}{:}
             \PY{l+s+sd}{\PYZdq{}\PYZdq{}\PYZdq{}Sweep a range of values for beta.}
         \PY{l+s+sd}{    }
         \PY{l+s+sd}{    beta\PYZus{}array: array of beta values}
         \PY{l+s+sd}{    gamma: recovery rate}
         \PY{l+s+sd}{    }
         \PY{l+s+sd}{    returns: SweepSeries that maps from beta to total infected}
         \PY{l+s+sd}{    \PYZdq{}\PYZdq{}\PYZdq{}}
             \PY{n}{sweep} \PY{o}{=} \PY{n}{SweepSeries}\PY{p}{(}\PY{p}{)}
             \PY{k}{for} \PY{n}{init\PYZus{}im} \PY{o+ow}{in} \PY{n}{imun\PYZus{}rate\PYZus{}array}\PY{p}{:}
                 \PY{n}{system} \PY{o}{=} \PY{n}{make\PYZus{}system}\PY{p}{(}\PY{n}{pop}\PY{p}{,} \PY{n}{init\PYZus{}im}\PY{p}{,} \PY{n}{beta}\PY{p}{,} \PY{n}{gamma}\PY{p}{,} \PY{n}{sigma}\PY{p}{)}
                 \PY{n}{results} \PY{o}{=} \PY{n}{run\PYZus{}simulation}\PY{p}{(}\PY{n}{system}\PY{p}{,} \PY{n}{update\PYZus{}func}\PY{p}{)}
                 \PY{n}{sweep}\PY{p}{[}\PY{n}{system}\PY{o}{.}\PY{n}{init\PYZus{}im}\PY{p}{]} \PY{o}{=} \PY{n}{calc\PYZus{}highest\PYZus{}infected}\PY{p}{(}\PY{n}{results}\PY{p}{)}\PY{o}{*}\PY{n}{pop}
             \PY{k}{return} \PY{n}{sweep}
\end{Verbatim}


    To carefully check out the impact due to the change of initial immunity
for the community, a sweep\_init\_im function was created. The function
checks out the highest number of people infected to the disease during
the simulation. Since the number of people being infected at a day is
proportional to the number of currently infected people, higher numbers
means that the disease is spreading faster.

    \begin{Verbatim}[commandchars=\\\{\}]
{\color{incolor}In [{\color{incolor}75}]:} \PY{n}{imun\PYZus{}rate\PYZus{}array} \PY{o}{=} \PY{n}{linspace}\PY{p}{(}\PY{l+m+mi}{0}\PY{p}{,} \PY{l+m+mi}{1}\PY{p}{,} \PY{l+m+mi}{21}\PY{p}{)}
         \PY{n}{sweep} \PY{o}{=} \PY{n}{sweep\PYZus{}init\PYZus{}im}\PY{p}{(}\PY{n}{imun\PYZus{}rate\PYZus{}array}\PY{p}{)} 
         \PY{n}{sweep}
\end{Verbatim}


\begin{Verbatim}[commandchars=\\\{\}]
{\color{outcolor}Out[{\color{outcolor}75}]:} 0.00    208.313630
         0.05    193.245208
         0.10    177.918762
         0.15    162.651083
         0.20    147.602195
         0.25    132.761627
         0.30    117.985077
         0.35    103.523808
         0.40     89.378869
         0.45     75.521163
         0.50     62.132854
         0.55     49.290309
         0.60     37.160085
         0.65     25.952817
         0.70     15.994027
         0.75      7.777191
         0.80      2.115958
         0.85      1.000000
         0.90      1.000000
         0.95      1.000000
         1.00      0.999000
         dtype: float64
\end{Verbatim}
            
    \begin{Verbatim}[commandchars=\\\{\}]
{\color{incolor}In [{\color{incolor}76}]:} \PY{n}{plot}\PY{p}{(}\PY{n}{sweep}\PY{p}{)}
         \PY{n}{decorate}\PY{p}{(}\PY{n}{xlabel}\PY{o}{=}\PY{l+s+s1}{\PYZsq{}}\PY{l+s+s1}{Immunity Rate}\PY{l+s+s1}{\PYZsq{}}\PY{p}{,}
                 \PY{n}{ylabel} \PY{o}{=} \PY{l+s+s1}{\PYZsq{}}\PY{l+s+s1}{Highest number of people infected during 1 outbreak}\PY{l+s+s1}{\PYZsq{}}\PY{p}{,}
                 \PY{n}{title} \PY{o}{=} \PY{l+s+s1}{\PYZsq{}}\PY{l+s+s1}{Figure 3}\PY{l+s+s1}{\PYZsq{}}\PY{p}{)}
\end{Verbatim}


    \begin{center}
    \adjustimage{max size={0.9\linewidth}{0.9\paperheight}}{output_26_0.png}
    \end{center}
    { \hspace*{\fill} \\}
    
    Looking at the table and the plot, we can examine that the speed of
infection decreases almost linearly until the immunity rate reachs 80\%.
Actually, the table states that the maximum number of people infected
after the initial immunization rate of 85\% is 1, meaning that no one
except for the initially infected person was infected during the
outbreak. We guessed that the herd immunity for measles in this
simulation must be around 80\textasciitilde{}85\% range.

    \begin{Verbatim}[commandchars=\\\{\}]
{\color{incolor}In [{\color{incolor}77}]:} \PY{k}{def} \PY{n+nf}{calc\PYZus{}fraction\PYZus{}infected}\PY{p}{(}\PY{n}{results}\PY{p}{)}\PY{p}{:}
             \PY{l+s+sd}{\PYZdq{}\PYZdq{}\PYZdq{}Fraction of susceptible population infected during the simulation.}
         \PY{l+s+sd}{    }
         \PY{l+s+sd}{    results: DataFrame with columns S, E, I, R}
         \PY{l+s+sd}{    }
         \PY{l+s+sd}{    returns: fraction of susceptible group population}
         \PY{l+s+sd}{    \PYZdq{}\PYZdq{}\PYZdq{}}
         
             \PY{k}{return} \PY{p}{(}\PY{n}{get\PYZus{}first\PYZus{}value}\PY{p}{(}\PY{n}{results}\PY{o}{.}\PY{n}{S}\PY{p}{)} \PY{o}{\PYZhy{}} \PY{n}{get\PYZus{}last\PYZus{}value}\PY{p}{(}\PY{n}{results}\PY{o}{.}\PY{n}{S}\PY{p}{)}\PY{p}{)}\PY{o}{/}\PY{n}{get\PYZus{}first\PYZus{}value}\PY{p}{(}\PY{n}{results}\PY{o}{.}\PY{n}{S}\PY{p}{)}
\end{Verbatim}


    \begin{Verbatim}[commandchars=\\\{\}]
{\color{incolor}In [{\color{incolor}78}]:} \PY{k}{def} \PY{n+nf}{sweep\PYZus{}init\PYZus{}im2}\PY{p}{(}\PY{n}{imun\PYZus{}rate\PYZus{}array}\PY{p}{)}\PY{p}{:}
             \PY{l+s+sd}{\PYZdq{}\PYZdq{}\PYZdq{}Sweep a range of values for beta.}
         \PY{l+s+sd}{    }
         \PY{l+s+sd}{    beta\PYZus{}array: array of beta values}
         \PY{l+s+sd}{    gamma: recovery rate}
         \PY{l+s+sd}{    }
         \PY{l+s+sd}{    returns: SweepSeries that maps from beta to total infected}
         \PY{l+s+sd}{    \PYZdq{}\PYZdq{}\PYZdq{}}
             \PY{n}{sweep} \PY{o}{=} \PY{n}{SweepSeries}\PY{p}{(}\PY{p}{)}
             \PY{k}{for} \PY{n}{init\PYZus{}im} \PY{o+ow}{in} \PY{n}{imun\PYZus{}rate\PYZus{}array}\PY{p}{:}
                 \PY{n}{system} \PY{o}{=} \PY{n}{make\PYZus{}system}\PY{p}{(}\PY{n}{pop}\PY{p}{,} \PY{n}{init\PYZus{}im}\PY{p}{,} \PY{n}{beta}\PY{p}{,} \PY{n}{gamma}\PY{p}{,} \PY{n}{sigma}\PY{p}{)}
                 \PY{n}{results} \PY{o}{=} \PY{n}{run\PYZus{}simulation}\PY{p}{(}\PY{n}{system}\PY{p}{,} \PY{n}{update\PYZus{}func}\PY{p}{)}
                 \PY{n}{sweep}\PY{p}{[}\PY{n}{system}\PY{o}{.}\PY{n}{init\PYZus{}im}\PY{p}{]} \PY{o}{=} \PY{n}{calc\PYZus{}fraction\PYZus{}infected}\PY{p}{(}\PY{n}{results}\PY{p}{)} \PY{o}{*} \PY{l+m+mi}{100} 
             \PY{k}{return} \PY{n}{sweep}
\end{Verbatim}


    To do a deeper analysis, another sweep\_init\_im function was created to
check out the percentage of people in the susceptible group infected
during the outbreak. It will give us more clear view toward the herd
immunity for measles and hopefully reveal the danger of lowering
immunity rate for a community.

    \begin{Verbatim}[commandchars=\\\{\}]
{\color{incolor}In [{\color{incolor}79}]:} \PY{n}{imun\PYZus{}rate\PYZus{}array} \PY{o}{=} \PY{n}{linspace}\PY{p}{(}\PY{l+m+mi}{0}\PY{p}{,} \PY{l+m+mf}{0.99}\PY{p}{,} \PY{l+m+mi}{34}\PY{p}{)}
         \PY{n}{sweep2} \PY{o}{=} \PY{n}{sweep\PYZus{}init\PYZus{}im2}\PY{p}{(}\PY{n}{imun\PYZus{}rate\PYZus{}array}\PY{p}{)} 
         \PY{n}{sweep2}
\end{Verbatim}


\begin{Verbatim}[commandchars=\\\{\}]
{\color{outcolor}Out[{\color{outcolor}79}]:} 0.00    99.880178
         0.03    99.851143
         0.06    99.813900
         0.09    99.767596
         0.12    99.710066
         0.15    99.638624
         0.18    99.549939
         0.21    99.439865
         0.24    99.303238
         0.27    99.133598
         0.30    98.922845
         0.33    98.660774
         0.36    98.334471
         0.39    97.927495
         0.42    97.418780
         0.45    96.781122
         0.48    95.979083
         0.51    94.966023
         0.54    93.679834
         0.57    92.036695
         0.60    89.921705
         0.63    87.174520
         0.66    83.566653
         0.69    78.764043
         0.72    72.258928
         0.75    63.203631
         0.78    49.789004
         0.81    28.661237
         0.84     9.100615
         0.87     3.086305
         0.90     1.642769
         0.93     1.105863
         0.96     0.831930
         0.99     0.666378
         dtype: float64
\end{Verbatim}
            
    \begin{Verbatim}[commandchars=\\\{\}]
{\color{incolor}In [{\color{incolor}80}]:} \PY{n}{plot}\PY{p}{(}\PY{n}{sweep2}\PY{p}{)}
         \PY{n}{decorate}\PY{p}{(}\PY{n}{xlabel}\PY{o}{=}\PY{l+s+s1}{\PYZsq{}}\PY{l+s+s1}{Immunity Rate}\PY{l+s+s1}{\PYZsq{}}\PY{p}{,}
                 \PY{n}{ylabel} \PY{o}{=} \PY{l+s+s1}{\PYZsq{}}\PY{l+s+si}{\PYZpc{} o}\PY{l+s+s1}{f susceptible people getting measles during an outbreak}\PY{l+s+s1}{\PYZsq{}}\PY{p}{,}
                 \PY{n}{title} \PY{o}{=} \PY{l+s+s1}{\PYZsq{}}\PY{l+s+s1}{Figure 4}\PY{l+s+s1}{\PYZsq{}}\PY{p}{)}
\end{Verbatim}


    \begin{center}
    \adjustimage{max size={0.9\linewidth}{0.9\paperheight}}{output_32_0.png}
    \end{center}
    { \hspace*{\fill} \\}
    
    Until the immunity rate reaches 60\%, more than 90\% of people in the
susceptible group is infected by the measles. However, the percentage
drops abruptly after that, hitting less than 10\% on immunity rate of
84\%. This graph clearly shows the importance of herd immunity, and the
threat people can face due to the lowering of the immunity rate.

    \hypertarget{results}{%
\subsection{Results}\label{results}}

This model uses SEIR methodology to examine how measels would spread
throughout a community of 1000 individuals with varying immunity rates.
Figure 1 depicts an SEIR representation based on a 90.7\% measles
immunity rate, equivalent to that of the immunity rate in the United
States. Due to the high immunity rate, susceptible people are protected
by the herd immunity, and the number of individuals in each of the
categories, susceptible, recovered, and infected remains constant
throughout the simulated outbreak.

Figure 2 represents an example of the SEIR model with an immunity rate
of 30\%. In this model, we can see that as the number of susceptible
individuals decreases, the number of recovered individuals increases at
an equal and opposite rate. The entire population get infected and later
recovered from this measles outbreak within 150 days of the start.

Figure 3 depicts the predicted outcome of this model that as the
immunity rate in a community increases, rate of infection decreases,
thus the number of people infected during an outbreak will decrease. We
see the number of infected individuals plateau around
80\%\textasciitilde{}85\% immunity.

Figure 4 depicts the percent of susceptible individuals that do contact
measles during an outbreak. At low immunity rates (without herd
immunity) a large percent of susceptible individuals do contact measles.
As the immunity rate increases, this percentage decreases.

    \hypertarget{interpretation}{%
\subsection{Interpretation}\label{interpretation}}

As expected, as the immunity rate in the community increased, the
highest number of people infected with measles during an outbreak
decreased. The number of people infected with measles begins to plateau
between an 80 - 85\% immunity rate. From the data that Figure 4 is based
on we can see that the ideal immunity rate for a community should be
more than 80 - 85\%, because the herd immunity is lost at the lowered
immunity rate. Between these 2 numbers, the percent of susceptible
individuals that contract measles drops sharply from 36\% to 6\%.

Our model does have several limitations: 1. We were unable to find an
effective contact number or contact rate for measles within the United
States. Having this number would have enabled us to calculate beta
instead of just assuming it to be 0.9.

\begin{enumerate}
\def\labelenumi{\arabic{enumi}.}
\setcounter{enumi}{1}
\item
  The model gets to a point where less than 1 person is infected with
  measles. This is physically impossible as you cannot have less than
  one person. In our results, we interpreted less than 1 to mean the
  individual did not have measles.
\item
  The outbreak usually happens country wide, not restricted into a
  single community. Due to the fact that the simulation was done in a
  close community, the results may vary in real world situation.
\item
  People who get measles are usually quarantined before they start
  infecting other people. One special feature about measles is the rash,
  which usuaully appears 14 days after exposure. In real world, people
  get quarantined right away when they get the rash. In this simulation,
  the factor was ignored. People can also get a MMR vaccination while
  they are exposed, meaning that not every exposed people move to the
  infected stage.
\item
  Measles spread differently among different age groups. Usually, it
  spread easily among the younger children. The age factor was ignored
  in this simulation due to its complexity.
\end{enumerate}

    \hypertarget{abstract}{%
\subsection{Abstract}\label{abstract}}

In this model, we were seeking to find out the result of lowering the
measles immunity rate in a small community during a outbreak. As
predicted, we found that as the immunity rate in a community is lowered,
the number of infections in a community increases. We also found that
when immunity is between 80-85\%, the number of individuals infected in
a population begins to plateau. This finding indicated that the ideal
immunity rate for a community of 1000 individuals is between 80-85\%.

    \begin{Verbatim}[commandchars=\\\{\}]
{\color{incolor}In [{\color{incolor}81}]:} \PY{n}{plot}\PY{p}{(}\PY{n}{sweep}\PY{p}{)}
         \PY{n}{decorate}\PY{p}{(}\PY{n}{xlabel}\PY{o}{=}\PY{l+s+s1}{\PYZsq{}}\PY{l+s+s1}{Immunity Rate}\PY{l+s+s1}{\PYZsq{}}\PY{p}{,}
                 \PY{n}{ylabel} \PY{o}{=} \PY{l+s+s1}{\PYZsq{}}\PY{l+s+s1}{Highest number of people infected during 1 outbreak}\PY{l+s+s1}{\PYZsq{}}\PY{p}{,}
                 \PY{n}{title} \PY{o}{=} \PY{l+s+s1}{\PYZsq{}}\PY{l+s+s1}{Figure 3}\PY{l+s+s1}{\PYZsq{}}\PY{p}{)}
\end{Verbatim}


    \begin{center}
    \adjustimage{max size={0.9\linewidth}{0.9\paperheight}}{output_37_0.png}
    \end{center}
    { \hspace*{\fill} \\}
    
    \begin{Verbatim}[commandchars=\\\{\}]
{\color{incolor}In [{\color{incolor}82}]:} \PY{n}{plot}\PY{p}{(}\PY{n}{sweep2}\PY{p}{)}
         \PY{n}{decorate}\PY{p}{(}\PY{n}{xlabel}\PY{o}{=}\PY{l+s+s1}{\PYZsq{}}\PY{l+s+s1}{Immunity Rate}\PY{l+s+s1}{\PYZsq{}}\PY{p}{,}
                 \PY{n}{ylabel} \PY{o}{=} \PY{l+s+s1}{\PYZsq{}}\PY{l+s+si}{\PYZpc{} o}\PY{l+s+s1}{f susceptible people getting measles during an outbreak}\PY{l+s+s1}{\PYZsq{}}\PY{p}{,}
                 \PY{n}{title} \PY{o}{=} \PY{l+s+s1}{\PYZsq{}}\PY{l+s+s1}{Figure 4}\PY{l+s+s1}{\PYZsq{}}\PY{p}{)}
\end{Verbatim}


    \begin{center}
    \adjustimage{max size={0.9\linewidth}{0.9\paperheight}}{output_38_0.png}
    \end{center}
    { \hspace*{\fill} \\}
    

    % Add a bibliography block to the postdoc
    
    
    
    \end{document}
